% Format heavily inspired by 9a from https://www.rpi.edu/dept/arc/training/latex/resumes/
\documentclass[margin]{res}
\setlength{\textwidth}{5.1in}

\begin{document}

 \moveleft.5\hoffset\centerline{\huge\bf Nick Torres}
 \moveleft\hoffset\vbox{\hrule width\resumewidth height 1pt}\smallskip
 \moveleft.5\hoffset\centerline{\texttt{nickrtorres@icloud} - \texttt{github.com/nickrtorres}}

\begin{resume}

\section{Education}
                    {\sl Master of Science,} Computer Science \\
                    California State University, Long Beach, expected Fall 2021 \\
                    \\
                    {\sl Bachelor of Science,} Computer Science \\
                    California State University, Long Beach Spring 2018

\section{Skills}
                    {\sl Languages:} C++, Rust, python3, Java, shell (POSIX, bash), awk \\
                    {\sl Technology:} Git, Subversion, Docker, Google Test, Boost, Qt \\
                    {\sl Operating Systems:} Unix-like (GNU/Linux, macOS) \\

\section{Experience}
                {\sl Software Engineer} \hfill May 2018 -- July 2020 \\
                Safran Passenger Innovations, Brea, CA \\
                Worked with a team to design, develop, document
                and test software for an inflight entertainment system.
                 \begin{itemize}
                    \item Wrote correct libraries and applications in C++.
                    \item Tested software at the unit and integration levels.
                    \item Documented software at all levels from high level
                    overviews to API descriptions.
                    \item Maintained systems to build and deploy software.
                    \item Reviewed software, documentation, and tests written
                    by other engineers.
                \end{itemize}
\section{Open Source}
              {\sl \texttt{clippy} --- the Rust linter} \hfill \texttt{https://github.com/rust-lang/rust-clippy}
                 \begin{itemize}
                   \item Proposed and implemented a new lint to catch non-idiomatic use of
                   \texttt{Result}---a common Rust sum type.
                   \item Worked through the pull request process with the
                   project maintainers to integrate the new lint into
                   \texttt{clippy}.
                   \item Contributed documentation fixes and updates.
                 \end{itemize}
\section{Projects}
             {\sl \texttt{rlox} --- a Lox interpreter} \hfill \texttt{https://github.com/nickrtorres/rlox} \\
             A work-in-progress port of \texttt{jlox}---Bob Nystrom's Lox interpreter
             written for {\sl Crafting Interpreters} in Java.
                 \begin{itemize}
                   \item Ported the Java implementation of Lox's four-stage,
                   tree walk interpreter to idiomatic Rust.
                   \item Developed a test fixture that takes advantage of
                   Rust's macro system to improve future extensibility.
                 \end{itemize}

             {\sl \texttt{program} --- a Rust port of \texttt{perror(3)}} \hfill \texttt{https://crates.io/crates/program}
                 \begin{itemize}
                   \item Developed a library that emulates \texttt{perror(3)}'s
                   error handling in idiomatic Rust.
                   \item Published crate on crates.io that can be pulled in as
                   a dependency on any Rust project.
                   \item Wrote documentation that explains the intricacies of
                   the API and sets the expectations of developers.
                 \end{itemize}
\end{resume}
\end{document}
