% Format heavily inspired by 9a from https://www.rpi.edu/dept/arc/training/latex/resumes/
\documentclass[margin]{res}
\setlength{\textwidth}{5.1in}

\begin{document}

 \moveleft.5\hoffset\centerline{\huge\bf Nick Torres}
 \moveleft\hoffset\vbox{\hrule width\resumewidth height 1pt}\smallskip
 \moveleft.5\hoffset\centerline{\texttt{nickrtorres@icloud.com} -- \texttt{github.com/nickrtorres} -- \texttt{(714) 454 5058}}

\begin{resume}

\section{Experience}
                {\sl Instructional Student Assistant} \hfill November 2020 --
                April 2021 \\
                California State University, Long Beach -- Long Beach, CA \\
                Assisted undergraduate computer science students with topics
                from classes throughout the undergraduate computer science
                curriculum -- topics included programming languages, operating
                systems, and analysis of algorithms.

                {\sl Software Engineer / Product Developer} \hfill May 2018 -- July 2020 \\
                Safran Passenger Innovations -- Brea, CA\\
                Worked with a team to design, develop, document, and test
                software for RAVE, a complete inflight entertainment system
                including servers and hundreds of clients.
                 \begin{itemize}
                    \item Wrote libraries, applications, and automated tests in C++.
                    \item Created small utilities in python3 and shell to
                    increase productivity.
                    \item Documented software at all levels from high level
                    overviews to API descriptions.
                    \item Reviewed software, documentation, and tests written
                    by other engineers.
                    \item Developed interview procedures for future employees.
                    \item Participated in all events of the Agile software development methodology.
                    \item Attended certified scrum developer (CSD) training to
                    learn the intricacies of test driven development and the
                    importance of automated tests for developing maintainable
                    software.
                \end{itemize}

\section{Education}
                    {\sl Bachelor of Science,} Computer Science \\
                    California State University, Long Beach, Spring 2018

\section{Skills}
                    {\sl Languages:} C++\{14, 17\}, python3, Rust, shell (POSIX, bash) \\
                    {\sl Technology:} Git, Subversion, Docker, Google Test  \\
                    {\sl Operating Systems:} \textit{Unix-like} (GNU/Linux, macOS) \\

\section{Open Source}
              {\sl \texttt{clippy}} \hfill \texttt{https://github.com/rust-lang/rust-clippy/pull/5415}
                 \begin{itemize}
                   \item Proposed and implemented a new lint to catch a non-idiomatic use of
                   \texttt{Result}--a common Rust sum type.
                   \item Worked through the pull request process with the
                   project maintainers to integrate the new lint into
                   \texttt{clippy}.
                 \end{itemize}

\section{Projects}
             {\sl \texttt{Bandera}} \hfill \texttt{https://github.com/nickrtorres/bandera} \\
             A portable x86-16 bytecode virtual machine that accepts a small
             subset of \textit{MASM} flavored assembly language and DOS
             \texttt{int 21h} interrupts -- namely those that perform I/O. It
             uses a recursive descent parser to understand the semantics of a
             \textit{MASM} assembly program and emits bytecode for that file to
             run on its virtual machine.

             {\sl \texttt{rlox}} \hfill \texttt{https://github.com/nickrtorres/rlox} \\
             A rust port of \texttt{jlox}--Bob Nystrom's Lox interpreter
             written for {\sl Crafting Interpreters}.

\end{resume}
\end{document}
